

\chapter{Overview}
\setcounter{page}{1}

\section{Introduction}
Implementing research pulse sequences on GE MR scanners requires EPIC programming, a time-consuming and error-prone task with a steep learning curve.
Moreover, pulse sequences need to be recompiled after each scanner software upgrade, which is sometimes problematic.

This user guide describes the ``toppe.e'' pulse sequence for GE scanners, which allows the entire sequence to be specified with a set of external files created with a high-level software tool such as Matlab.
This makes it possible to play arbitrary sequences of RF pulses and gradient waveforms, which enables \textbf{rapid prototyping of sequences without the need for low-level EPIC programming}.
With \toppe, the task of pulse programming a GE scanner becomes one of creating the various external files that define the sequence.

\myfigure{overview}{0.95}{
Overview of pulse sequence programming with \toppe.
The TOPPE sequence files (orange) specify all details of the MR acquisition, such as RF phase cycling, gradient waveforms, and timing information.
These files are created in MATLAB using the TOPPE MATLAB toolbox (\url{https://github.com/toppeMRI/matlab/}).
The TOPPE sequence files are passed on to the TOPPE interpreter (green), a binary executable that executes the sequence on the scanner.
%Source code (EPIC) for the interpreter is available for download on the GE Collaboration Community site at \url{https://collaborate.mr.gehealthcare.com/docs/DOC-1247}.
The interpreter only needs to be compiled and installed once per scanner software upgrade.
With this setup, arbitrary sequences of RF and gradient pulses can be played out, which enables rapid pulse sequence prototyping without the need for EPIC programming.
}

\toppe~was developed as a research tool at the fMRI laboratory at University of Michigan, and has to date been used in several projects including stack-of-spirals imaging, Bloch-Siegert B1+ mapping, echo-shifted RF-spoiled imaging (PRESTO), steady-state imaging with 3D tailored RF excitation, and dual-echo steady-state (DESS) imaging.

We are currently making \toppe~compatible with {\bf Pulseq}, an open file format for compactly describing MR sequences.
See Chapter~\ref{ch:pulseq} for more information about using \toppe~as a GE ``interpreter module'' for Pulseq files.

\textbf{Your feedback} is most welcome. 
%Finally, if you are interesting in joining the development team, we'd love to hear from you.


\section{Required external files}
In addition to the {\tt toppe} and {\tt toppe.psd.o} executables, the following files are needed to run a particular scan (see Fig.~\ref{fig:files}):

\subsection{ {\tt $\star$.mod} files}
\toppe~creates several unique ``cores'' (or modules), with each core/module associated with one .mod file (Fig.~\ref{fig:files}).
For example, an RF excitation module may be defined by a file called {\tt tipdown.mod} that specifies the RF amplitude and phase waveforms (rho and theta) and all three gradients.
Similarly, a Cartesian (spin-warp) gradient-echo readout may be defined in a file {\tt readout.mod} that contains readout and phase-encode gradient waveforms.
Finally, a spoiler gradient can be defined in a file {\tt spoiler.mod}.
Each .mod file is unique up to waveform scale factors and to a rotation in the logical xy-plane, and typically only a few .mod files are needed.
Note that each .mod file gives rise to a separate {\tt createseq()} call in \toppe.
Also, each .mod file can contain multiple waveform shapes, that can be selected dynamically (column \waveformnum~in {\tt scanloop.txt}).

\subsection{\tt modules.txt}
The various $\star$.mod files needed to define a scan are listed in a small text file named {\tt modules.txt}, which simply contains a line for each .mod file specifying the file name, core duration, and whether it is an RF excitation module, readout module, or gradients-only module.
Values are tab-separated.
A {\tt modules.txt} file for our simple spin-warp imaging example may look like this:
\begin{lstlisting}
Total number of unique cores
3
wavfile_name	duration(us)	hasRF?	hasDAQ?
tipdown.mod	0	1	0
readout.mod	0	0	1
spoiler.mod	0	0	0
\end{lstlisting}
A duration of 0 means that the minimum core duration for that .mod file will be used.

\subsection{\tt scanloop.txt}
Finally, the complete MR scan loop is specified in {\tt scanloop.txt}, in which each line corresponds to a separate {\tt startseq()} call in \toppe.
Each line in scanloop.txt must contain the following tab-separated values:

\begin{tabular}{ l l l}
  \textbf{column} \# & \textbf{entry} & \textbf{units/type} \\
  \modulenum	& module number 								& positive integer, starting at 1 \\
  \rfamp    	& rf waveform (rho) amplitude				& signed even short int16 (-32766 to +32766) \\
  \thetaamp 	& phase waveform (theta) amplitude		& signed even short int16 (-32766 to +32766) \\
  \gxamp			& Gx waveform amplitude						& signed even short int16 (-32766 to +32766) \\
  \gyamp			& Gy waveform amplitude						& signed even short int16 (-32766 to +32766) \\
  \gzamp			& Gz waveform amplitude						& signed even short int16 (-32766 to +32766) \\
  \dabslice		& data storage 'slice' index				& positive integer, starting at 1 \\ 
  \dabecho		& data storage 'echo' index				& positive integer, starting at 0 \\ 
  \dabview		& data storage 'view' index				& positive integer, starting at 1 \\ 
  \dabmode		& turn on/off data acquisition			& one of two integers: 0 (off) or 1 (on) \\
  \inplanerot	& in-plane (x-y) rotation angle			& signed even short int16: -32766 (-pi rad) to +32766 (+pi rad) \\
  \txphase		& RF transmit phase							& signed even short int16: -32766 (-pi rad) to +32766 (+pi rad) \\
  \recphase		& receive phase								& signed even short int16: -32766 (-pi rad) to +32766 (+pi rad) \\
  \textra		& time added to end of module				& positive integer, in microseconds \\
  \rffreq		& RF transmit frequency offset			& integer, in Hz \\
  \waveformnum	& waveform number 							& positive integer, starting at 1
\end{tabular} \\ [0.2in]

\begin{comment}
\newcommand\modulenum{1}
\newcommand\rfamp{2}
\newcommand\thetaamp{3}
\newcommand\gxamp{4}
\newcommand\gyamp{5}
\newcommand\gzamp{6}
\newcommand\dabslice{7}
\newcommand\dabecho{8}
\newcommand\dabview{9}
\newcommand\dabmode{10}
\newcommand\inplanerot{11}
\newcommand\txphase{12}
\newcommand\recphase{13}
\newcommand\textra{14}
\newcommand\rffreq{15}
\newcommand\waveformnum{16}
\end{comment}

Example: A {\tt scanloop.txt} file for single-slice, RF-spoiled spin-warp imaging with 256 phase-encodes might begin like this:
\begin{lstlisting}
nt	maxslice	maxecho	maxview
768	1	0	768	
Core iarf iath iagx iagy iagz slice echo view dabon rot rfph recph textra freq
1	32766	32766	0	0	32766	0	0	0	0	0	0	0	0	0	1
2	0	0	32766	32766	-32638	1	0	1	1	0	0	0	0	0	1
3	0	0	0	0	32766	0	0	0	0	0	0	0	0	0	1
1	32766	32766	0	0	32766	0	0	0	0	0	21298	0	0	0	1
2	0	0	32766	32766	-32382	1	0	2	1	0	0	21298	0	0	1
3	0	0	0	0	32766	0	0	0	0	0	0	0	0	0	1
...
\end{lstlisting}
where nt is the total number of {\tt startseq()} calls (256 phase-encodes $\times$ 3 cores per TR), and \texttt{maxslice}, \texttt{maxecho}, and \texttt{maxview} correspond to the maximum values of slice, echo, and view, respectively. %\texttt{rhnslices}$-$1, \texttt{rhnecho}$-$1, and \texttt{rhnframes}, respectively, in \toppe.
%\textbf{We do not recommend using the {\tt slice=0} and {\tt view=0} indeces (``slots'')}, hence the {\tt slice} and {\tt view} indeces both start at 1 in the above example.
Values are tab-separated.
For long scans, {\tt scanloop.txt} can contain many tens of thousands of lines.

%tipdowncore ia_tipdown ia_th max_pg_iamp max_pg_iamp max_pg_iamp dabslice dabecho dabview daboff 0 irfphase irfphase textra rffreq waveform

%Figure~\ref{fig:overview} summarizes the ...


\myfigure{files}{0.95}
{
Overview of the TOPPE file structure.
A TOPPE MR sequence is comprised of a (usually) small number of unique modules that are listed in the file \texttt{'modules.txt'} (a).
Each module contains a set of arbitrary gradient waveforms and a complex RF waveform (if applicable), and is contained within a file with extension \texttt{`.mod'}.
These modules are unique up to waveform scale factors, and to a rotation in the logical transverse (k$_\textrm{xy}$) plane.
(b) Example of four different modules within an MR sequence: two 3D RF excitations, one spiral-in readout, and one gradient spoiler.
(c) The file \texttt{'scanloop.txt'} lists the order in which to execute the modules, and specifies all other dynamic sequence information.
Each row specifies the module number as listed in \texttt{modules.txt}; RF and gradient waveform amplitudes;
where to store the acquired data in memory ('slice/echo/view' indices).
in-plane rotation angle ('phi');
transmit and receive phase;
a parameter 'textra' by which to extend module duration (for dynamic TR changes);
and RF transmit frequency (for slice offsets).
For detailing sequence timing information, see Ch.~\ref{ch:timing}.
}


\section{Source code and other resources}

TOPPE is open-source and is available at the following sites:

\subsection{https://github.com/toppeMRI/matlab}

Matlab scripts for creating and viewing TOPPE files. Also contains complete sequence examples.
To access the code, you can either browse the website, or copy the entire repository to local disk as follows:

\begin{lstlisting}
git clone https://github.com/toppemri/matlab
\end{lstlisting}


\subsection{Binary executable (driver/interpreter}

See \url{https://toppemri.github.io/} for more info.


\subsection{https://github.com/toppeMRI/UserGuide}

Latex source files for this user guide.
To access, browse the website or copy the entire repository to local disk as follows:
\begin{lstlisting}
git clone https://github.com/toppemri/UserGuide
\end{lstlisting}



\subsection{https://toppemri.github.io/}

The TOPPE website.




% EOF
